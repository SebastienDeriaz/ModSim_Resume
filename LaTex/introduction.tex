\documentclass[resume]{subfiles}

\begin{document}
\section{Introduction}

\subsection{Algèbre linéaire}

\subsubsection{Indépendance linéaire}

Les vecteurs sont linéairement indépendants si il n'existe pas de combinaison linéaire nulle.

Ou alors on regroupe ces vecteur dans un matrice est on calcul son déterminant si il est non nulle alors tous les vecteurs sont linéairement indépendants.

\subsubsection{rang}
Son rang est égal au nombre de colonnes de la matrice qui sont linéairement indépendantes.
$rank(A) = rank(A^T)$ 
donc le rang ne peut pas être plus grand que la plus petite des dimension de la matrice NxM.

\subsubsection{Bases}

Une base de $E^n$ est un ensemble de n vecteurs linéairement indépendants.
La base standard est l'ensemble des vecteurs de base (x,y,z)($e_1,e_2,e_3$)

Tous les vecteurs sont des combinaisons linéaires des vecteurs de base 
$x=x_1\cdot u_1 + x_2\cdot u_2+...+x_n\cdot u_n$ 

Le changement de base aura pour effet de changer les éléments du vecteur 
$x = z_1\cdot p_1 + z_2\cdot p-2 +...+ z_n\cdot p_n$ 

\paragraph{Formule changement de base}


$x=Pz$ et $z=P^{-1}x$ , la matrice P correspond au regroupement des vecteurs de la nouvelle base

$P = [p_1|p_2] @ p_1=\begin{bmatrix} p_{11}\\ p_{12}\end{bmatrix},p_2=\begin{bmatrix} p_{21}\\ p_{22}\end{bmatrix}$ $P= \begin{bmatrix} 
p_{11}&p_{21}\\
p_{12}&p_{22}
\end{bmatrix}$ 

La matrice $A$ dans la base $u$ est équivalente à la matrice
$P^{-1}AP$ dans la base $p$  

\subsubsection{Vecteurs propres}

$Ax = \lambda x$ 

les valeurs propres sont calculées à partir de la formule $det(A-I\lambda) = 0$ 

\subsubsection{Matrice modale}

$M =[e_1|e_2|...|e_n]$ 

La transformation de A dans la nouvelle base est $\Lambda=M^{-1}AM$ 

Exemple $M=\begin{bmatrix} 1 & 1 \\ -1 & 2 \end{bmatrix}$ 

\subsection{Equations différentielles}

EDO vs Equation aux différences

$\frac{dy(t)}{dt} = ay(t) \rightarrow y(t+1)=(a+1)y(t)$ 

\subsubsection{Exemple équation aux différences}

$y(k + n) + a_{n-1}y(k + n-1) +...+ a_0y(k) = 0$

$\lambda^{k+n} + a_{n-1}\lambda^{k+n-1} + . . . + a_0\lambda^k = 0$ 

$\lambda^{n} + a_{n-1}\lambda^{n-1} + . . . + a_0\lambda = 0$ 

\subsubsection{Exemple EDO}
\begin{multline*}
\frac{dy}{dt} = ay \rightarrow y(t) = Ce^{at}\\
y(0) = Ce^{a\cdot 0} = Ce^0 = C\\
y(t) = y(0)e^{at}\\
\frac{d^2y}{dt^2} + \omega^2y = 0\\
my'' = -ky\\
\lambda^2 + \frac{k}{m}=0 \rightarrow \lambda = \pm j\sqrt{\frac{k}{m}}=\pm j\omega\\
y(t)=C_1 e^{j\omega t}+C_2 e^{-j\omega t} = Asin(\omega t)+Bcos(\omega t)
\end{multline*}


\end{document}