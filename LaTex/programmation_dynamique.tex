\documentclass[resume]{subfiles}


\begin{document}
\section{Programmation dynamique}
Séparation d'un problème en étapes. Par exemple :
\begin{center}
Trouver $u(t)$ pour le système dynamique $\dot{x}(t)=f(x(t),u(t))$ qui maximise la fonction de coût $J(u(t))$ sur $t_0...t_f$ et respecte les contraintes
\end{center}
\subsection{Principe d'optimalité}
A partir de tout point d'une trajectoire optimale, la trajectoire restante est optimale pour le problème d'optimisation initialisé en ce point
\subsubsection{Optimal return function (ORF)}
$V(x,t)$ est la fonction de retour optimale. Dans un système à temps discret, on travaille à rebours pour trouver $V(x,k)$
$$V(x,k)=\max_{u\in U}\left[l(xu)+V(f(x,u),k+1)\right]$$
Dans le cas d'une allocation de ressources, on a 
$$X(x(N-1),N-1)=\sqrt{x(N-1)}$$




\end{document}