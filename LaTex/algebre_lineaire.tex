\documentclass[resume]{subfiles}


\begin{document}
\section{Algèbre linéaire}
\subsection{Indépendance linéaire}
$$\alpha_1 v_1+\alpha_2v_2+\cdots+\alpha_mv_m=0\qquad \alpha_i\neq 0$$
Pour déterminer si les vecteurs sont linéairement indépendants on construit la matrice $A$
$$A=\begin{bmatrix}
\\
v_1 & v_2 & \cdots & v_n\\
\\
\end{bmatrix}$$
$$\det(A)\neq 0\longrightarrow \text{ linéairement indépendants}$$
$$\text{rang}(A)=N_\text{colonnes}\longrightarrow \text{ linéairement indépendants}$$
\subsection{Bases}
Une base $E^n$ est un ensemble de $n$ vecteurs linéairement indépendants. Chaque vecteur est une somme de combinaison linéaire des vecteurs de base
$$x=\sum_{i=1}^{n} x_iu_i$$
\subsubsection{Changement de base}
$$\text{base }U=\begin{bmatrix}
1 & 0\\
0 & 1
\end{bmatrix}\qquad \text{base }E=\begin{bmatrix}
3 & -1\\
1 & 1
\end{bmatrix}$$
La matrice $P$ est constituée des vecteurs de la nouvelle base
$$P=\begin{bmatrix}
3 & -1\\
1 & 1
\end{bmatrix}$$

\begin{align*}
\text{base }U\longrightarrow &\text{ base }E & x_E&=P^{-1}x_U\\
\text{base }E\longrightarrow &\text{ base }U & x_U&=Px_E
\end{align*}

La matrice $A$ dans la base $U$ est équivalente à la matrice
$P^{-1}AP$ dans la base $E$  
\subsubsection{Changement de base d'une matrice}
$$B=P^{-1}AP$$
\subsection{Valeurs propres}
les valeurs propres $\lambda$ sont les solutions de l'équation
$$\det\left(A-\lambda I\right)=\vec{0}$$
On cherche les solutions de l'équation
$$\boxed{Ax=\lambda x}\longleftrightarrow \boxed{(A-\lambda I)x=\vec{0}}$$
La multiplicité numérique d'une valeur propre est son exposant dans le polynôme caractéristique.
\subsubsection{Vecteurs propres}
On trouve les vecteurs propres $\vec{x}$ avec
$$\left(A-\lambda_i I\right)\vec{x}_i=\vec{0}$$
Sur python on a :\\
\verb!val_propres, vect_propres = np.linalg.eig(A)!\\
Les vecteurs propres sont linéairement indépendants
\subsection{Matrice modale}
C'est la matrice formée par les vecteurs propres d'une matrice
$$M=\begin{bmatrix}
\\
v_1 & v_2 & \cdots & v_n\\
\\
\end{bmatrix}$$
\subsubsection{Diagonalisation de $A$}
Les vecteurs propres de $A$ constituent une nouvelle base. $\Lambda$ est "l'opération" de $\Lambda$ dans cette nouvelle base
$$\Lambda=M^{-1}AM$$
\end{document}