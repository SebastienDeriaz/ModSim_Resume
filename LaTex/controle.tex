\documentclass[resume]{subfiles}

\begin{document}
\section{Concept de contrôle-commande}

\subsection{Exemple du train}
\begin{itemize}
\item Système: le train
\item Variable contrôlée: la vitesse (suivi de trajectoire)
\item Trajectoire désirée: fixée en temps réel par le conducteur
\item Perturbations: variations de charges (passagers) et de profil (déclivité de la route, …)
\item Variable manipulée: couple aux roues  
\end{itemize}

\subsection{La loi de commande}
\begin{itemize}
\item La loi de commande peut être
  \subitem continue
  \subitem discrète (on/off)
  \subitem basée sur les évènements
\item La loi de commande peut être implantée
  \subitem manuellement
  \subitem automatique
    \subsubitem analogique
    \subsubitem digitale   
\end{itemize}

\subsection{Système de contrôle}
\begin{itemize}
\item Le contrôleur adapte la variable de commande (manipulée), pour atteindre la valeur désirée pour la variable contrôlée
\item Il y a deux classes principales de stratégie de contrôle
  \subitem feedforward (anticipation) ou open-loop (boucle ouverte)
  \subitem feedback (rétroaction) ou closed-loop (boucle fermée)
\item Parfois les deux sont implantées simultanément (FB/FF)
  \subitem FF traite le rejet de perturbation et/ou l’anticipation du chat de consigne
  \subitem FB cible le suivi de trajectoire
  \subitem FB/FF très fréquent en chemical engineering  
\end{itemize}

\subsection{Commande}
\begin{itemize}
\item Boucle Ouverte

La loi de commande est déterminée indépendamment de
la valeur de la variable contrôlée

\item Boucle Fermée
\end{itemize}

La commande dépend de la valeur de la grandeur
contrôlée

\end{document}