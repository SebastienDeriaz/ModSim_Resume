\documentclass[resume]{subfiles}

\begin{document}
\section{Système non-linéaire}

Nous ne pouvons pas utilisé les outils d'analyse classique 
\begin{itemize}
\item temps discret : $x(t+1)=f(x(t),t)$ 
\item temps continu : $\dot{x}(t)=f(x(t),t)$ 
\end{itemize}

Si le système est LTI alors f n'a pas de dépendance en temps

\subsection{Équilibre}

On les obtient en résolvant
\begin{itemize}
\item $\bar{x}=f(\bar{x},t)$ (cas discret)
\item $0=f(\bar{x},t)$ (cas continu)
\end{itemize}

Mais les choses ne sont pas si simples pour les sys. NL
\begin{itemize}
\item Résoudre les équations n'est pas trivial
\item Les systèmes peuvent avoir 1, aucun, de nombreux points d'équilibre.  
\end{itemize}
\subsection{Lyapunov indirect}

La méthode consiste à étudier le système dans le voisinage d'un point équilibre 

Si la zone est suffisamment petite alors le système NL peut être approché par un développement en série de Taylor au 1er ordre (linéaire)

$f(\bar{x}+\delta x(t))= f(\bar{x})+[\frac{\partial f}{\partial x}]_{\bar{x}} \delta x(t)$ 

\subsection{Le Jacobien}

$$
F_J=\begin{pmatrix}
\frac{\partial f_1}{\partial x_1} & \frac{\partial f_1}{\partial x_2} & \cdots & \frac{\partial f_1}{\partial x_n} \\
\frac{\partial f_2}{\partial x_1} & \frac{\partial f_2}{\partial x_2} & \cdots & \frac{\partial f_2}{\partial x_n} \\
\vdots & \vdots & \ddots & \vdots\\
\frac{\partial f_n}{\partial x_1} & \frac{\partial f_n}{\partial x_2} & \cdots & \frac{\partial f_n}{\partial x_n} \\
\end{pmatrix}
$$

 Linéarisation avec le Jacobien

\subsubsection{Temps discret}

$\delta \bar{x}(t) = F_J \delta x(t)$ 

\subsubsection{Temps continu}

$\delta\bar{x}(t) = F_J\delta x(t)$ 

\subsection{Lyapunov direct}

S'il existe une fonction de Lyapunov $V(x)$ dans une boule $S(\bar{x},R_0)$ de centre $\bar{x}$, alors le point d'équilibre $\bar{x}$ est stable. Si, de plus, $\dot{V}(x)$ est strictement négatif en tout point (sauf $\bar{x}$), alors la stabilité est asymptotique.  

Pas dans l'examen trop complexe sur papier.

\end{document}