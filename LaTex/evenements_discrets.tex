\documentclass[resume]{subfiles}


\begin{document}
\section{Simulation à événements discrets}
\begin{center}
\small
\begin{tabular}{ll}
Temps moyen d'attente dans la file & $E[W_k]$\\
Temps d'attente maximum dans la file & $W_k$\\
Temps moyen dans le système & $E[S_k]$\\
Temps maximum passé dans le système & $S_k$\\
Nombre moyen de clients dans la file & $E[X(t)]$\\
Nombre maximum de clients dans la file & $X(t)$\\
\end{tabular}
\end{center}
\subsection{Entités}
Les entités sont les objets dynamiques de la simulation (clients, pièces, tâches, etc...)
\subsection{Attributs}
Les attributs sont les caractéristiques communes des entités (par exemple la quantité que chaque client veut acheter)
\subsection{Ressources}
Les ressources représentes les "ingrédients" qu'utilisent les entités pour réaliser leurs tâches
\subsection{File}
Place d'attente lorsqu'une entité ne peut pas saisir une ressource
\subsection{Accumulateurs statistiques}
Par exemple : nombre total d'entités, temps total d'attente, etc...
\subsection{Événements}
Quelque chose (arrivée, départ, etc...) qui arrive à un instant $t$ qui peut changer :
\begin{itemize}
\item des attributs
\item des variables
\item des accumulateurs statistiques
\end{itemize}



\end{document}