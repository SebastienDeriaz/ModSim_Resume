\documentclass[resume]{subfiles}


\begin{document}
\section{Observabilité}

\subsection{Matrice d'observabilité}
$$S^T=\begin{bmatrix}
C & CA & CA^2 & \cdots & CA^{n-1}
\end{bmatrix}^T=\begin{bmatrix}
C^T & A^TC^T & (A^2)^TC^T & \cdots & (A^{n-1})^TC^T
\end{bmatrix}$$
$$\boxed{\det\left(S^T\right)=\det\left(S\right)\neq 0\longrightarrow\text{ Observable}}$$
Un système est observable si il est possible de déterminer son état initial en observant sa sortie (en un temps fini depuis le temps initial).\\
Pour introduire un observateur, il faut que le système soit observable.




\subsection{Définition}

Le système $\dot{x}(t) = Ax (t)$, $y = Cx(t)$   est complètement observable s'il existe un temps fini $t^* > 0$ tel que la  connaissance de $y (t)$  sur $[0\space t^*]$est suffisante pour déterminer la valeur de l'état initial $x(0)$

\subsection{Théorème}

Un système à temps continu (discret) est complètement observable si et  seulement si la matrice d'observabilité: $S^T = [C, CA, CA^2 ,..., CA^{n-1}]^T = [C^T, A^TC^T , (A^2)^T C^T,..., (A^{n-1})^T C^T]$ 

 est de rang n (rang plein)

\subsection{Observateur}

Un observateur est un système dynamique qui retourne une estimation de la valeur de l'état quand on le 'nourrit' avec les sorties mesurées.

\subsubsection{Observateur trivial (copie)}

Pour un système $\dot{x}(t)=Ax(t)+Bu(t)$

L'observateur est $\dot{z}(t)= Az(t)+Bu(t)$

Mais avec cette observateur l'erreur $\varepsilon(t)=z(t)-x(t)$ donc $\dot{\varepsilon}(t)=A\varepsilon(t)$ l'erreur disparait que lorsque le système est stable.

\subsubsection{Observateur identité}
\begin{figure}[H]
\centering
\includegraphics[width=0.9\columnwidth]{img_23.png}
\end{figure}

\paragraph{Système} :
$$\begin{cases}\dot{x}(t)=Ax(t)+Bu(t)\\y(t)=Cx(t)\end{cases}$$
\paragraph{Observateur} :
$$\begin{cases}\dot{z}(t)=Az(t)+E\left[y(t)-Cz(t)\right]+Bu(t)\\
u(t)=Kz(t)\end{cases}$$
On choisi $E$ pour déterminer la dynamique de l'observateur et $K$ pour la dynamique du régulateur
\paragraph{Dynamique de l'erreur} $\dot{\varepsilon}(t) = \dot{z}(t) - \dot{x}(t) = [A - EC] (z(t) - x(t)) = [A - EC] \varepsilon(t)$ 
\paragraph{Dynamique de l'observateur (boucle fermée)} $A-EC$
\paragraph{Dynamique du régulateur (boucle fermée)} $A-BK$



\begin{itemize}
\item Si $z(0)=x(0)$ alors $z(t)=x(t)$ pour tout $t>0$ 
\item Si $z(0)\neq x(0)$ le vecteur d'erreur est gouverné par $[A-EC]$ 
\item On peut placer les valeurs propres de cette matrice, avec le degré de liberté que constitue E  
\end{itemize}
$A-EC$ donne la dynamique de l'observateur (qu'on va utiliser pour faire un placement de pôle de l'observateur). Attention ! La fonction place ou acker va déterminer $K$ dans $A-BK$. On doit donc faire
\verb!acker(A.T, C.T).T! (voir \ref{transpose_matrice})
\subsubsection{Observateur d'ordre réduit}

Un observateur d’ordre réduit peut être construit, pour que l'effort soit sur les variables "inconnues"

  Pour un système \begin{equation}\begin{cases}\dot{x}(t)=Ax(t)+Bu(t)\\y(t)=Cx(t)\end{cases}\end{equation} avec (A, C) complètement observable et C(pxn) de rang p

Voir les pages 14-16 du polycopier 2\_11 Observability 

\subsection{Contrôleur stabilisant}

\begin{figure}[H]
    \centering
    \includegraphics[width=1\columnwidth]{Figures/CtrlStab_1.png}
\end{figure}

\subsubsection{Théorème}

 Pour un système \begin{equation}\begin{cases}\dot{x}(t)=Ax(t)+Bu(t)\\y(t)=Cx(t)\end{cases}\end{equation} avec l'observateur identité $\dot{z}(t)= Az(t)+E[y(t)-Cz(t)]+Bu(t)$ et la loi de commande $u(t)=Kz(t)$ 

Le polynôme caractéristique de ce composite est égal au produit des
polynômes caractéristiques de $A+BK$ et de $A-EC$:  

$\Delta_{A+BK}(\lambda)\cdot \Delta_{A-EC}(\lambda)$  
\begin{itemize}
\item Les matrices $K$ et $E$ peuvent être fixées indépendamment
\item Ce théorème s'applique aux systèmes linéaires  
\end{itemize}

\end{document}