\documentclass[resume]{subfiles}

\begin{document}
\section{Outils}
\subsection{Matrices}
\paragraph{Multiplication matricielle}
$$\underset{\textcolor{RoyalBlue}{a}\times \textcolor{OrangeRed}{b}}{A}\cdot \underset{\textcolor{OrangeRed}{b}\times \textcolor{ForestGreen}{c}}{B}=\underset{\textcolor{RoyalBlue}{a}\times \textcolor{ForestGreen}{c}}{C}$$
\subsubsection{Transposition}
\begin{enumerate}
\item $(A+B)^T=A^T+B^T$
\item $(AB)^T=B^TA^T$
\end{enumerate}

\begin{equation}
\boxed{A-BK=\left(A^T-K^TB^T  \right)^{T}}
\label{transpose_matrice}
\end{equation}
\subsection{Linéarisation}
Pour linéariser une fonction $f$ au point $x_0$ on effectue
$$f'(x_0)(x-x_0)+f(x_0)$$
Dans le cas multivariable on a
\begin{multline*}
f(x,y) \approx f(x_0,y_0) + \left. {\frac{{\partial f(x,y)}}{{\partial x}}} \right|_{x_0,y_0} \\
(x - x_0) + \left. {\frac{{\partial f(x,y)}}{{\partial y}}} \right|_{x_0,y_0} (y - b_0)
\end{multline*}

\end{document}